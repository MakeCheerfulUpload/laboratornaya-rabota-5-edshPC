\documentclass[a4paper]{article}
\usepackage[a4paper, includefoot,
left=1cm, right=1cm, top=1.5cm, bottom=1.5cm,
headsep=0.5cm, footskip=0.5cm]{geometry}%правильные поля
\usepackage[english, russian]{babel}
\usepackage[utf8]{inputenc}
\usepackage{multicol}%3 колонки
\usepackage{graphicx}%для картинок
\usepackage{float}%для картинок
\graphicspath{{./images/}}%путь картинок
\usepackage{tabularx}%работа с таблицами для подзаголовков
\usepackage{ragged2e}%выравнивание по ширине
\usepackage{titlesec}%отступы в заголовках
\usepackage{latexsym}
\titlespacing\section{0pt}{12pt plus 4pt minus 2pt}{0pt plus 2pt minus 2pt}
\titlespacing\subsection{0pt}{12pt plus 4pt minus 2pt}{0pt plus 2pt minus 2pt}
%Оформление подзаголовков
\renewcommand{\subsubsection}[1]{\flushleft
\begin{tabularx}{\linewidth}{X}
\hline \\ [-27pt]
\textbf{\flushleft\sffamily\large #1} \\
\hline \\ [-7pt]
\end{tabularx}
\justifying}
\newcommand{\n}{\textit{n}}%быстрый набор курсивной n
\renewcommand\tabularxcolumn[1]{m{#1}} %Для первой буквы по центру
\newcommand{\allpage}[1]{{\resizebox{\linewidth}{!}{#1}}}%размер текста на всю строку


\begin{document}
\pagestyle{empty} % убираем страницы
\flushright
\textbf{\sffamily ISSN 0130$\cdot$2221}
\flushleft
\textbf{\sffamily\LARGE НОЯБРЬ/ДЕКАБРЬ
\hfill
1994$\cdot$№6}
\center
\allpage{КВАНТ} \\
\vspace{0.5cm} 
\allpage{\textbf{\sffamily ФИЗИКО-МАТЕМАТИЧЕСКИЙ ЖУРНАЛ ДЛЯ ШКОЛЬНИКОВ И СТУДЕНТОВ}} \\
\vspace{0.5cm}
\includegraphics[width=0.9\linewidth]{titul}
\newpage
\pagestyle{plain} % возвращаем страницы
\section*{\center\Huge О правильных многоугольниках,\\функции Эйлера и числах Ферма}
\setcounter{page}{15} %15 страница
\subsection*{\center\sffamily А.КИРИЛЛОВ}
\begin{multicols}{3}
\subsubsection{Пролог}

%Первая буква З
\flushleft
\begin{tabularx}{\linewidth}
{>{\hsize=1\hsize\linewidth=\hsize}X
>{\hsize=14.75\hsize\linewidth=\hsize}X}
\\ [-25pt] %правильный отступ
\textbf{\fontsize{50}{60}\selectfont З} & АДАЧИ на геометрические построения — один из самых популярных в школьной математике.
\end{tabularx}
\justifying

Почти в каждом математическом
кружке разбираются такие задачи.
Это, конечно, не случайно. История
геометрических построений насхитывает несколько тысяч лет, и уже
древние греки достигли здбоь большого искусста. В качестве примера
можно привести задачу Аполлония:
\textit{построить окружность, касающуюся трех данных окружностей.}

Многим, вероятно, известны три
знаменитые задачи древности, оказавшнеся неразрешимыми: \textit{о кваратуре круга, трисекции угла и удвоении куба.}

Но, пожалуй, самой красивой является задача о построении правильных многоугольников. Собственно говоря, это не одна задача, а целая
серня задач: \textit{для каждого натурального числа $n \ge 3$ требуется с помощью циркуля и линейкы построить правильный n-угольник}

Для некоторых значений \n эта задача совеем простая (например, для \n = 3, 4, 6, 8, 12); для других — посложнее ( \n = 5, 10, 15; ниже мы расскажем, как построить десятиугольник и пятиугольник); для третьих — очень
сложная ( \n = 17 или 257). Наконец
существуют такие значения \n, для
которых эта задача вообще неразрешима (например, \n = 7, 9, 11).

Выпишем подряд несколько натуральных чисел, начиная \n = 3,
отметим красным цветом те числа \n,
для которых можно построить правильный \n-угольних циркулем и линейкой:
\begin{figure}[H]
  \includegraphics[width=\linewidth]{numbers}
  \caption{Необходимые числа}
\end{figure}
\footnotetext{Эта статья впервые была опубликована\\в "Кванте" № 7 за 1977 год.}

Есть ли какая-нибудь закономерность в распределении «красных» и
«черных» чисел? Оказывается, есть;
но найти ее довольно трудно. Эта закономерность имеет арифметическую
природу; чтобы се описать, нам придется временно оставить геометрию и
заняться элементами теории чисел —
высшего раздела арифиетики.

\subsubsection{Функция Эйлера}

Важной арифметической харакристекой числа \n является количество чисел, меньших \n и взанино простых с \n.
Одним из первых это заметил
знаменитый математик XVIII века
Леонард Эйлер. Он препложил для
этого количества обозначение $\phi(n)$, и
с тех пор функция $n \to \phi(n)$ известна,
под именем «функции Эйлера». Например, для \n = 10 имеется четыре
числа, меньших десяти и взаимно простых с ним: 1, 3, 7 и 9; так что $\phi(10) = 4$.

Функция $\phi$ облалает многими интересными свойствами. Одно из них
было открыто еще самим Эйлером:
для любых двух взаимно простых
чисел \textit{m} и \n справедливо равенство:
$$
\phi(mn) = \phi(m)\phi(n). \eqno(1)
$$
Кроме того, легко проверить, что
\textit{если p — простое число, то}
$\phi(p) = p-1$,
$\phi(p^2) = p^2 - p$, \textit{и вообще}
$$
\phi(p^m) = p^{m-1}(p-1). \eqno(2)
$$
Эти свойства позволяют легко вычислять функцию Эйлера для небольших значений \n. Например,
$$
\phi(10) = \phi(2)\cdot\phi(5) = 1\cdot4 = 4,
$$\vspace{1.2cm}\\%отступ до таблиц
%Как не выравнивай колонки - всё равно криво :(
\begin{tabularx}{2.07\linewidth}{|>{\centering\arraybackslash\hsize=0.6cm}X|X|X|X|X|X|X|X|X|X|X|X|X|X|X|X|X|X|X|X|X|}
\hline
    \n & 1 & 2 & 3 & 4 & 5 & 6 & 7 & 8 & 9 & 10 & 11 & 12 & 13 & 14 & 15 & 16 & 17 & 18 & 19 & 20 \\
    \hline
    $\phi(n)$ & 1 & 1 & 2 & 2 & 4 & 2 & 6 & 4 & 6 & 4 & 10 & 4 & 12 & 6 & 8 & 8 & 16 & 6 & 18 & 8 \\
\hline
\end{tabularx}
\vspace{0.6cm}\\
\begin{tabularx}{2.07\linewidth}{|>{\centering\arraybackslash\hsize=0.6cm}X|X|X|X|X|X|X|X|X|X|X|X|X|X|X|X|X|X|X|X|X|X|X|}
\hline
    \n & 21 & 22 & 23 & 24 & 25 & 26 & 27 & 28 & 29 & 30 & 31 & 32 & 33 & 34 & 35 & 36 & 37 & 38 & 39 & 40 & 41 & 42 \\
    \hline
    $\phi(n)$ & 12 & 10 & 22 & 8 & 20 & 12 & 24 & 12 & 28 & 8 & 30 & 16 & 20 & 16 & 24 & 12 & 36 & 18 & 24 & 16 & 40 & 12 \\
\hline
\end{tabularx}
\vfill\null\columnbreak
$$
\phi(100) = \phi(4)\cdot\phi(25) = 2\cdot20 = 40.
$$
Мы приводим адесь значения функции Эйлера для \n от 1 до 42 (см.
таблицы 1, \textit{а} и \textit{б}).

Сравните эти таблицы с приведенным выше рядом «красных» и «черных» чисел. Не правда ли, связь между «цветом» чисел \n и значением
$\phi(n)$ уже легко утодывается? Мы
видим, что если правильный \n-угольник можно построить с помощью
циркуля и линейки, то соответствующее значение функции $\phi(n)$ является
степенью двойки. Оказывается, это
условие является необходимым и
достаточным для возможности построення правильного \n-угольника.

В настоящей статье мы не сможем
строго доказать это. Однако мы приведем достаточно простые и убедительные соображения в пользу этого
факта. Аналогичные соображения
применимы и ко многим другим задачам на построение — например, к
залаче о трисекции угла.

\subsubsection{Что значит «построить»?}

Вопрос о точной постановке задач на
построение циркулем и линейкой уже
обсуждался на страницах «Кванта».
Мы не будем здесь еще раз предостерегать читателей от неправильного
употребления математических инструментов. Скажем лишь, что окончательное решение задачи на построение должно быть (хотя бы в принципе)
записываемо в виде цепочки элементарных операций, напоминающей систему команд, отдаваемых электронной вычислительной машине.
\flushright
\textit{Таблица 1, а}
\vspace{1.2cm}\\
\textit{Таблица 1, б}
\newpage
\end{multicols}
\flushleft
\huge
\begin{multicols}{2}
$\to$ - стремится

$\longrightarrow$ - отражение функции

$\Rightarrow$ - импликация

$\Longrightarrow$ - следование

$\hookrightarrow$ - включение одного множества в другое

$\mapsto$ - отображается в

$\longmapsto$ - отобр. с аргументами

$\gets$ - обратное стремление

$\longleftarrow$ - обр. отражение

$\Leftarrow$ - обр. импликация

$\Longleftarrow$ - обр. след-е

$\hookleftarrow$ - обр. включение

$\leftrightarrow$ - равнос-ть обл. опред. множеств

$\longleftrightarrow$ - равн-ть с аргументами

$\Leftrightarrow$ - равносильность

$\Longleftrightarrow$ - необх. и достаточность

$\uparrow$ - стремление вверх

$\Uparrow$ - импликация вверх

$\downarrow$ - стремление вниз

$\Downarrow$ - импликация вниз

$\updownarrow$ - равнос-ть ОП верт.

$\Updownarrow$ - равнос-ть верт.

$\nearrow$ - бинарное отношение

$\searrow$ - бинарное отношение

$\swarrow$ ...

$\nwarrow$

$\leftharpoondown$

$\leftharpoonup$

$\rightharpoonup$

$\rightharpoondown $ - бинарное отношение

$\rightleftharpoons$ - двойное бинарное отношение

$\leadsto$ - приводит к
\end{multicols}
\end{document}